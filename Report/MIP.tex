

\section{MIP Model}
The MIP formulation was developed with promo, a Python library for writing solver-independent models.
Two models were developed: a base version composed of a 4d array with symmetry breaking and implied constraints, and a better one, optimizing the number of variables featuring circle matching.

\subsection{Variables}
\subsubsection*{Decision Variables}
To encode the STS problem, two variables were used:
\begin{enumerate}
\item Y a matrix of (n-1)*(n//2) x (n*(n - 1)//2) binary values, the most strict possible formulation in order to minimize the variable count. The indices of the matrix are couples (w,p) and (i,j) where i<j
\item H a binary array of length (n*(n - 1)//2), indicating for each match (i,j) if the team i plays home or away  
The domains are [0,1] to ease the constraints formulation and to improve the search speed.
\end{enumerate}
\subsubsection*{Implied Variables}
The implied constraint needs a variable Zteam that keeps track of which team played on which period.
\subsubsection*{Objective Variables}
In order of minimizing the maximum imbalance we need 3 extra variables:
\begin{enumerate}
\item Home: the number of home matches played by team i
\item Away: the number of away matches played by team i
\item Z: the maximum imbalance itself, minimized by the objective function
\end{enumerate}
Despite having integers domains we can easily bound them to (0, n-1) massively reducing the search space.


\subsection{Constraints}
In this particular model the addition of symmetry breaking constraints did not improve the results, as circle matching already breaks some symmetries, this is why they are not included.

\subsubsection*{Necessary constraints}
\begin{enumerate}
\item \text{one\_match\_per\_period\_per\_week\_rule}: each week/period slot needs to have one match scheduled
\item \text{match\_scheduled\_once\_rule}: each match can be scheduled once for all week/period slots 
\item \text{max\_team\_match\_period}: a team k can play at most twice in the same period
\end{enumerate}


\subsubsection*{Implied constraints}
Especially useful in an optimization environment, this constraint aims to spread the matches of a team in different periods over the scheduling, to get a more balanced result. It also help with symmetry breaking.
\subsubsection*{Objective Constraints}
The \text{home\_games} and \text{away\_games} are meant to populate the variables Home and Away with the right values.
\text{balance\_max} sets both Home-Away <= Z and Away-Home <= Z for every team, to implement the objective.

\subsection{4d array model}
The simplest possible implementation of the STS problem, developed only for comparison as it is not nearly as efficient as the previous one.
The only decision variable is X, a 4d array n-1 x n/2 x n x n of binary values, where each cell represents a match, described as a combination of week, period, team1 and team2. The first team is the one playing home.
The constraints, apart from the necessary ones, include two symmetry-breaking constraints and two implied constraints.
The optimization is performed in the same way as in the previous model.

\subsection{Solvers}
The results were obtained by running the models for 3 different solvers: Cbc, Glpk and Guroby. Cbc and Glpk are open source solvers while Gurobi is proprietary, used under an academic license.

\subsection{Validation}
The model was implemented in Python using Z3's API. All experiments were carried out with respect to the given timeout of $300\,\mathrm{s}$ and with an increasing number of instances (i.e. number of teams, $n$).

In the following, we present tables and plots to compare the Z3 model using different encoding techniques, both with and without symmetry breaking constraints.

\subsubsection{Experimental Results (Decision version)}

\subsubsection{Experimental Results (Optimization version)}